\documentclass[a4paper,12pt]{article}
\usepackage[utf8]{inputenc}
\usepackage[T1]{fontenc}
\usepackage[provide=*,french]{babel}
\usepackage{lmodern}
\usepackage{titling}
\usepackage{hyperref}


\begin{document}

% page de garde
\begin{titlepage}
    \begin{center}
        \vspace*{2cm}
        {\huge Mini Socrate - Rapport de projet\par}
        \vspace{1cm}
        {\large \textit{Dans le cadre du cours INFO4030}\par}
        \vspace{2cm}
        {\large Alec Jones et Samuel Doucette\par}
        {\large A00216262\par}
        {\large A00216008\par}
    \end{center}
\end{titlepage}

\tableofcontents
\newpage

\section{Résumé}
Le programme repose sur trois fichiers principaux : lectureDesDonnees.cpp, menu.cpp et procedures.plsql. \\
\indent Le fichier lectureDesDonnees.cpp lit les fichiers CSV et remplit la base de données.
Le fichier menu.cpp offre une interface simple pour choisir et exécuter des requêtes.
Les procédures dans procedures.plsql gèrent la manipulation des données.

\section{Introduction}
Pour ce projet, nous avons été tâchés de créer une base de données qui
imite le fonctionnement de la base de données de l'Université de Moncton.
Pour ce faire, nous avons utilisé le langage de programmation C++ et l'API OCILIB ainsi que le plsql.

\section{Le Code}
\subsection{La lecture des données - lectureDesDonnees.cpp}
\indent Ce fichier utilise l'API C de OCILIB. Il a pour but d'initialiser la base de données en créant ses tables et en les remplissant avec les données des fichiers CSV fournis. \\
\indent Le programme résultant demande pour le nom d'utilisateur et le mot de passe à la base de données, afin que l'utilisateur puisse se brancher à son propre
compte. Une fois la connexion établie, le programme va effacer toutes les tables de la base, puis va les recréer à nouveau pour
qu'elles soient toutes vides. Ceci est fait avec les requêtes dans le fichier projet.sql dans Clic. \\
\indent Ensuite, le programme va lire les fichiers echelle.csv, evaluations.csv, programmes.csv et repertoire.csv pour remplir certaines tables.
À noter que ce ne sont pas toutes les tables qui sont peuplées par ce programme. Après, le programme ferme sa connexion au serveur puis se termine.

\subsection{Le PL/SQL - procedures.plsql}
Ce fichier contient les procédures PL/SQL qui sont utilisées par le programme menu.cpp. Il est divisé en plusieurs sections, chacune contenant des procédures
pour une tâche spécifique. \\
\indent Chaque procédure renvoie aussi des codes d'erreur, qui sont utilisés par le programme menu.cpp pour afficher des messages d'erreur à l'utilisateur. \\
\indent On est capable de compiler les procédures en exécutant le script.

\subsection{L'interface - menu.cpp}
\indent Ce fichier utilise aussi l'API C de OCILIB. Il agit comme l'interface entre l'utilisateur et la base de données, permettant
à l'utilisateur d'exécuter diverses requêtes avec l'aide de commandes simples dans la console. \\
\indent L'utilisateur doit en premier se brancher à la base de données en entrant son nom d'utilisateur et son mot de passe.
Dès que la connexion est établie, l'utilisateur aura accès à l'interface dans la ligne de commandes. L'utilisateur est demandé d'inscrire
une lettre qui correspond à diverses requêtes ou actions :
\begin{itemize}
    \item (P) - Afficher la liste des programmes, en ordre alphabétique.
    \item (C) - Afficher la liste des cours par session, en ordre alphabétique.
    \item (E) - S'inscrire comme nouvel.le étudiant.e à un programme existant ou bien se désinscrire.
    \item (I) - S'inscrire à un cours d'une certaine session ou bien se désinscrire.
    \item (N) - Ajouter ou supprimer une note obtenue à l'évaluation d'un cours à une session donnée.
    \item (F) - Calculer la note finale pour un cours d'une session donnée, si toutes les notes sont entrées.
    \item (M) - Calculer la moyenne cumulative d'une étudiante ou d'un étudiant.
    \item (H) - Afficher la liste de commandes disponibles.
    \item (X) - Sortir du programme.
\end{itemize}
\indent Dépendamment de la commande inscrite, l'utilisateur peut par la suite se voir demander de fournir de l'information supplémentaire,
comme un NI, un ID de programme, un sigle de cours, et ainsi de suite. La base de données est mise à jour selon les commandes exécutées.
Pour sortir du programme, l'utilisateur doit entrer la commande (X). \\
\indent Les requêtes sont faites avec les procédures définies dans procedures.plsql. Ces procédures doivent déjà être stockées dans la base de données
pour que le programme puisse les exécuter.

\subsection{Le fichier makefile - makefile}
Le fichier makefile est utilisé pour compiler le projet. Il contient les instructions pour compiler les fichiers .cpp et exécuter le programme.
Il est aussi utilisé pour rendre l'exécution et le développement du projet plus facile. \\
\indent Le makefile contient les instructions pour compiler le projet, exécuter le programme et nettoyer les fichiers temporaires.

\section{Conclusion}
Pour conclure, ce projet a été une bonne expérience d'apprentissage pour nous deux. Nous avons appris à utiliser l'API OCILIB, à écrire des procédures PL/SQL et à
interagir avec une base de données. \\
\indent C'était aussi une très bonne opportunité de pratiquer nos compétences qui en git et en makefile.
\end{document}